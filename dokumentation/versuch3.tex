% !TeX TXS-program:bibliography = txs:///biber

\documentclass[
    paper=a4,
    fontsize=10pt,
    DIV=12,
    % twocolumn,
    oneside,
]{scrartcl}

%___________________________________________________________________
% Imports
\input{preamble}

% \setlength{\imagewidth}{1\columnwidth}

%___________________________________________________________________
% Title Page
\title{Laborversuch zur Drehzahlregelung}
\labor{3}
\date{\today}

%___________________________________________________________________
% Begin
\begin{document}

\maketitle

%___________________________________________________________________
% Abstract
\begin{abstract}
    \noindent    
    \subsubsection*{Abstract}
        Im dritten Regelungstechniklabor wird das Verhalten eines Elektromotors mithilfe von Matlab und SIMULINK betrachtet. Es wird ein stationär genauer Regelkreis als PI-Regler erarbeitet und sein Verhalten bei Überschwingern und zeitabhängigen Lastmomenten beobachtet.
\end{abstract}

%___________________________________________________________________
% Text

\section{Motor im Nennbetrieb}
    Die Gleichungen für den Gleichstrommotor wurden der Laboranleitung \cite{versuch3} entnommen. 
    \begin{align}
        \omega  &= 73.68 \frac{\si{\newton\meter}}{\si{\ampere\kilogram\metre\squared}} \int i \, dt\\
        i       &= \frac{1}{\SI{1e8}{\henry}} \int u - 0.028 \si{\newton\meter\per\ampere} \omega - 5 \si{\ohm} i \, dt\\
        M       &= 0.028 \si{\newton\meter\per\ampere} i
    \end{align}
    
    In SIMULINK wird das zugehöirige Blockschaltbild als Subsystem entwickelt (siehe Abbildung \ref{fig:blockMotor}) und anschließend an eine konstante Spannungsversorgung von \SI{10}{\volt} angeschlossen. Der zeitliche Verlauf der Ausgangsgrößen ist in Abbildung \ref{fig:graphMotor} dargestellt. Die Zeitkonstante \(\tau\) berechnet sich zu \SI{2.3843}{\second}.

    \begin{figure}[hbt]
        \centering
        \includegraphics[width=1.3\imagewidth]{../versuch3/blockMotor}
        \caption{Blockschaltbild des Motor}
        \label{fig:blockMotor}
    \end{figure}   

    \begin{figure}[hbt]
        \centering
        \includegraphics[width=\imagewidth]{../versuch3/graphMotorModel.png}
        \caption{Motor im Nennbetrieb}
        \label{fig:graphMotor}
    \end{figure}   

\section{Motorsteuerung mit P-Regler}
    Um die Drehzahl des Motors einzustellen wird ein P-Regler erstellt und an das Motorsubsystem angeschlossen (siehe Abbildung \ref{fig:blockPController}). Die Eingangsspannung am Motor wird mit einem Sättigungsblock in SIMULINK auf \(U_{max} = \pm \SI{10}{\volt}\) begrenzt. Die Verstärkung \(K_R\) des Reglers wird so eingestellt, dass der Sättigungsblock gerade nicht begrenzt bei einem Sollwert von \(\omega_{soll} = \SI{100}{\radian\per\second}\).
    \begin{align}
        K_R = \frac{u_{max}}{\omega_{soll}} = \frac{\SI{10}{\volt}}{\SI{100}{\radian\per\second}} = \SI{0.1}{\volt\second\per\radian}
    \end{align}

    \begin{figure}[hbt]
        \centering
        \includegraphics[width=1.3\imagewidth]{../versuch3/blockPController}
        \caption{Blockschaltbild des P-Reglers}
        \label{fig:blockPController}
    \end{figure} 

    Der zeitliche Verlauf der Reglergrößen ist in Abbildung \ref{fig:graphPController} gezeigt. Es ist zu erkennen, dass der Motor nur \(\omega_{ist} = \SI{78,1}{\radian\per\second}\) erreicht und damit nicht stationär genau ist. Das ist darin begründet, dass in diesem Zustand die Eingangsspannung am Motor 
    \begin{align}
        U = K_R \cdot (\omega_{soll} - \omega_{ist}) = \SI{0.1}{\volt\second\per\radian} \cdot \left( \SI{100}{\radian\per\second} - \SI{78.1}{\radian\per\second} \right) = \SI{2.19}{\volt}
    \end{align}
    beträgt und diese Spannung wiederum zu einer Drehzahl von \(\omega_{ist} = \SI{78,1}{\radian\per\second}\) am Motor führt.

    \begin{figure}[hbt]
        \centering
        \includegraphics[width=\imagewidth]{../versuch3/graphPController}
        \caption{Reglergrößen des P-Reglers}
        \label{fig:graphPController}
    \end{figure}   

\section{Motorsteuerung mit PI-Regler}
    Der Regler aus dem vorherigen Abschnitt wird um einen Integrator erweitert (Abbildung \ref{fig:blockPiController}). Die beste Nachstellzeit \(T_N\), bei der gerade so kein Überschwinger für \(\omega_{soll} = \SI{100}{\radian\per\second}\) entsteht, wird mit einer ausgelagerten Funktion ermittelt (siehe Anhang \ref{lst:TN}). Dabei wird das Modell so lange mit veränderter Nachstellzeit aufgerufen, bis der Maximalwert der Reglergröße kleiner ist als der Sollwert. Es ergibt sich ein Wert von \(T_N = \SI{2.43}{\second}\)

    \begin{figure}[hbt]
        \centering
        \includegraphics[width=1.5\imagewidth]{../versuch3/blockPiController}
        \caption{Blockschaltbild des PI-Reglers}
        \label{fig:blockPiController}
    \end{figure}   

    Der zeitliche Verlauf der Reglergrößen ist in Abbildung \ref{fig:graphPiController} dargestellt. Zusätzlich wird die Spannung vor und nach dem Sättigungsblock betrachtet. Wird die Führungsgröße auf \(\omega_{soll} = \SI{300}{\radian\per\second}\) angehoben (Abbildung \ref{fig:graphPiControllerNew}), kommt es zu Überschwingern. Das liegt daran, dass in der Zeit, in welcher der Begrenzer aktiv ist, sich der I-Anteil des Reglers weiter erhöht. 

    \begin{figure}[hbt]
        \centering
        \begin{subfigure}{0.49\columnwidth}
            \includegraphics[width=1.0\columnwidth]{../versuch3/graphPiController.png}
            \caption{Sollwert \(\omega_{soll} = \SI{100}{\radian\per\second}\)}
            \label{fig:graphPiController}   
        \end{subfigure}%
        \hfill%
        \begin{subfigure}{0.49\columnwidth}
            \includegraphics[width=1.0\columnwidth]{../versuch3/graphPiControllerNew.png}
            \caption{Sollwert \(\omega_{soll} = \SI{300}{\radian\per\second}\)}
            \label{fig:graphPiControllerNew}
        \end{subfigure}
        \caption{Reglergrößen des P-Reglers}
    \end{figure}

\section{Motorregelung mit veränderlichen Lastmoment}
    Nun wird ein Lastmoment, dass der Drehbewegung entgegenwirkt, nach \SI{10}{\second} hinzugeschaltet. Das erweiterte Blockschaltbild ist in Abbildung \ref{fig:blockPiControllerLoad} zu finden, das Diagramm der Reglergrößen in Abbildung \ref{fig:graphPiControllerload}. Es wurde zusätzlich das Drehmoment des Motors und das Drehmoment am Ausgang aufgezeichnet.

    \begin{figure}[hbt]
        \centering
        \includegraphics[width=1.7\imagewidth]{../versuch3/blockPiControllerLoad}
        \caption{Blockschaltbild des PI-Reglers mit Lastmoment}
        \label{fig:blockPiControllerLoad}
    \end{figure}

    \begin{figure}[hbt]
        \centering
        \includegraphics[width=\imagewidth]{../versuch3/graphPiControllerLoad}
        \caption{Reglergrößen des PI-Reglers mit Lastmoment}
        \label{fig:graphPiControllerload}
    \end{figure}  

%___________________________________________________________________
% End

\printbibliography[heading=bibnumbered]

\section{Autorenbeiträge}
    Maileen Schwenk und Jan Hoegen erstellten die Vorbereitung und Messauswertung. Jan Hoegen schrieb den Bericht.

\section{Verfügbarkeit des Codes}
    Der Code zum Auswerten der Daten und Erstellen der Diagramme findet sich unter \url{https://github.com/JaxRaffnix/Regelungstechnik}. Ebenfalls ist hier der Code zum Erstellen dieser Ausarbeitung hinterlegt.

%___________________________________________________________________
% Appendix
\appendix   

\section{MATLAB-Code zur Modellauswertung}
    \lstinputlisting[language=MATLAB, label={lst:}]{../versuch3/plotMotorModel.m}
    \lstinputlisting[language=MATLAB, label={lst:}]{../versuch3/plotPController.m}
    \lstinputlisting[language=MATLAB, label={lst:}]{../versuch3/plotPiController.m}
    \lstinputlisting[language=MATLAB, label={lst:}]{../versuch3/plotPiControllerLoad.m}
    \lstinputlisting[language=MATLAB, label={lst:TN}]{../functions/BestResetTime.m}
    \lstinputlisting[language=MATLAB, label={lst:}]{../functions/TimeConstantIndex.m}

\end{document}