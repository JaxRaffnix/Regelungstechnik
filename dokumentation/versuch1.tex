% !TeX TXS-program:bibliography = txs:///biber

\documentclass[
    paper=a4,
    fontsize=10pt,
    DIV=13,
    % twocolumn,
    oneside,
]{scrartcl}

%-------------------------------------------------------------------------------------------------
%   Packages, Style Guide
%-------------------------------------------------------------------------------------------------

\input{preamble}

%-------------------------------------------------------------------------------------------------
%   Title Page
%-------------------------------------------------------------------------------------------------

\labor{1}
\date{\today}

%-------------------------------------------------------------------------------------------------
%   Begin
%-------------------------------------------------------------------------------------------------

\begin{document}

\maketitle

%-------------------------------------------------------------------------------------------------
%   Abstract
%-------------------------------------------------------------------------------------------------

\begin{abstract}
    \noindent    
    \subsubsection*{Abstract}
        In diesem Laborbericht werden grundlegende Funktionen von MATLAB verwendet, um Systeme zu beschreiben, zu analysieren und grafisch darzustellen. Im ersten Abschnitt werden    Sinussignale und ihre Lissajous-Figuren im Zeitbereich dargestellt. Anschließend wird ein Hochpass erster Ordnung simuliert und durch sein Bodediagramm und seine Ortskurve dargestellt. Abschließend wird die Temperaturregelung eines Backofens betrachtet. Mit SIMULINK wird ein Blockschaltbild erzeugt, damit werden die Regelvariablen simuliert und abgebildet.
\end{abstract}

%-------------------------------------------------------------------------------------------------
%   Text
%-------------------------------------------------------------------------------------------------

\section{Sinussignale im Zeitbereich}
    Die Funktionen \(x_1(t)\), \(x_2(t)\) und \(x_3(t)\) mit:

    \begin{align}
        x_1(t) &= 2 \cdot sin(2\pi \cdot \SI{2}{\kilo\hertz} \cdot t)\\
        x_1(t) &= 2 \cdot sin(2\pi \cdot \SI{6}{\kilo\hertz} \cdot t - \frac{\pi}{4})\\
        x_3(t) &= x_1(t) \cdot x_1(t)
    \end{align}

    \noindent
    aus der Versuchsanleitung \cite{versuch1} werden für den Zeitbereich \SIrange{0}{3}{\milli\second} und \num{e3} Abtastpunkten mit MATLAB simuliert und in Abbildung \ref{fig:sinus} dargestellt. Aus dem Diagramm lässt sich ablesen, dass die Frequenz von \(x_3(t)\) genau das doppelte der Frequenz von \(x_1(t)\) mit einem DC-Offset ist.
    Darüber hinaus wird eine Lissajous-Figur erstellt. Dabei werden auf den Diagrammachsen die Funktionen \(x_1(t)\) (x-Achse) und \(x_2(t)\) (y-Achse) eingetragen. Es entsteht ein geschlossenes Muster, da das Frequenzverhältnis mit 1:3 rational ist. 

    \begin{figure}[hbt]
        \centering
        \includegraphics[width=\imagewidth]{../versuch1/sinus}
        \caption{Darstellung der Sinussignale}
        \legend{\(10^3\) Abtastpunkte}
        \label{fig:sinus}
    \end{figure}

    Wird der Zeitbereich der Lissajous-Figuren jedoch auf \SIrange{0}{3}{\second} bei gleicher Anzahl an Abtastpunkten gelegt, ändert sich die Abtastrate \(f_s\) von 

    \begin{align}
        f_s = \frac{\num{e3}}{\SI{3e-3}{\second}} = \SI{333.3}{\kilo\hertz}
        \intertext{zu der neuen Frequenz}
        f_{s,neu} = \frac{\num{e3}}{\SI{3}{\second}} = \SI{333.3}{\hertz}
    \end{align}

    Die Frequenz der Signale \(x_1(t)\) und \(x_2(t)\) ist deutlich größer als die halbe Abtastfrequenz \(f_{s,neu}\), dass Abtasttheorem von Shannon ist verletzt und es kommt zum Aliasing. Da das Frequenzverhältnis der falsch aufgenommen Signale weiterhin identisch ist, bleibt die Lissajous-Figur identisch.
    
    Beim Verändern der Abtastpunkte entsteht ein nicht interpretierbares Bild. Das ist darin begründet, dass das Frequenzverhältnis der gemessenen Frequenzen nicht mehr als Bruch zweier ganzer Zahlen dargestellt werden kann. Beide Änderungen sind in Abbildung \ref{fig:lissajous} gezeigt. Der Code zum Erstellen der Grafiken befindet sich im Anhang \ref{lst:sinus}.

    \begin{figure}[hbt]
        \centering
        \includegraphics[width=\imagewidth]{../versuch1/lissjaou}
        \caption{Fehlerhafte Lissajous-Figuren}
        \label{fig:lissajous}
    \end{figure}

\section{Analyse eines Hochpasses}
    Zunächst werden die Bauteilwerte eines RC-Hochpasses bestimmt, bevor die Analyse mit MATLAB beginnt. Allgemein gilt für den Zusammenhang zwischen Eingangsspannung \(U_e\) und Ausgangsspannung \(U_e\) bei einem RC-Hochpass erster Ordnung:
    
    \begin{align}
        \label{eq:hochpass}
        \frac{U_a}{U_e} &= \frac{jwRC}{1+jwRC}
        \intertext{Und die Grenzfrequenz berechnet sich mit}
        f_g &= \frac{1}{2 \pi R C}\\
        \intertext{Wird der Widerstand \(R\) auf \SI{1}{\kilo\ohm} und die Grenzfrequenz zu \SI{e5}{\hertz} gewählt, berechnet sich die Kapazität zu:}
        C &= \frac{1}{2 \pi R f_g} = \SI{1.59e-9}{\farad}
    \end{align}

    Nun kann das vollständige Übertragungssystem in MATLAB verwendet werden. Das erzeugte Bodediagramm findet sich in Abbildung \ref{fig:bode} und die zugehörige Ortskurve in Abbildung \ref{fig:ortskurve}. Da beim Betrachten von negativen Frequenzen auf der Ortskurve die Funktion achsensymmetrisch zur x-Achse ist, kann das Diagramm ohne den Verlust von Informationen um genau diese Spiegelung verkürzt werden. Dies wird durch die Option \verb|ShowFullContour='off'| des \verb|nyquistplot|-Befehls erreicht. Der Code zum Erstellen der Diagramme findet sich in Anhang \ref{lst:tiefpass}.

    \begin{figure}[hbt]
        \centering
        \includegraphics[width=\imagewidth]{../versuch1/bode}
        \caption{Bodediagramm des Hochpasses}
        \label{fig:bode}
    \end{figure}

    \begin{figure}
        \centering
        \includegraphics[width=\imagewidth]{../versuch1/ortskurve}
        \caption{Ortskurven des Hochpasses}
        \label{fig:ortskurve}
    \end{figure}

    \subsection{Nachweis der Kreisfunktion}

    Die Abbildung \ref{fig:ortskurve} zeigt einen Kreis mit Radius 0.5 und dem Mittelpunkt \complexnum{0.5 +0i}. Folgende Rechnung bringt den Nachweis, dass es sich tatsächlich um einen Kreis handeln muss..

    Für einen Kreis mit Mittelpunkt \(m\), sowie Radius \(r\), gilt in der komplexen Zahlenebene mit der Variablen \(z \in \mathbb{C}\):
    
    \begin{align}
        \left| z - m \right| = r
    \end{align}

    Einsetzen der Gleichung \eqref{eq:hochpass} für \(z\) und \(m= 0.5 +0i\) ergibt:

    \begin{align}
        \left|\frac{jwRC}{1+jwRC} - \complexnum{0.5 +0i} \right| = \left| \frac{1}{2} \cdot \frac{-1 + jwRC}{1 + jwRC} \right| = \left| \frac{1}{2} \cdot \frac{(wRC)^2 - 1}{(wRC)^2 + 1} + i \frac{wRC}{(wRC)^2 + 1} \right| \\
        = \sqrt{ \frac{1}{4} \left( \frac{(wRC)^2 - 1}{(wRC)^2 + 1} \right)^2 + \frac{1}{4} \left( \frac{2wRC}{(wRC)^2 + 1} \right)^2} = \frac{1}{2} \sqrt{ \frac{ \left(\left(wRC\right)^2 - 1\right)^2 + 4 (wRC)^2 }{\left(\left(wRC\right)^2 + 1\right)^2} } = \frac{1}{2}
    \end{align}

    Somit ist gezeigt, dass der Radius tatsächlich 0.5 beträgt und die Kreisgleichung erfüllt ist.

\section{Temperaturregler mit SIMULINK}
    Zuletzt wird der Temperaturregler für einen Backofen betrachtet. Aus der Aufgabenstellung \cite{versuch1} wird das Blockschaltbild in SIMULINK erzeugt (siehe Abbildung \ref{fig:tempregler_block}). Der Puls-Generator am Eingang simuliert das Einschalten auf eine Solltemperatur von \SI{160}{\celsius} zu Beginn der Simulation und das Ausschalten nach \SI{30}{\minute}. Mit dem Parameter \(K_1\) wird diese Temperatur zu einem Spannungswert übersetzt. Der Zweipunktregler schaltet bei der Spannung \(u_{ein}\) ein und gibt eine Spannung von \SI{230}{\volt} aus. Bei einer Eingangsspannung von \(u_{aus}\) schaltet er ab. Das Heizgerät wird mit einem PT1-Glied simuliert. Dabei wird mit \(K_2\) das Ergebnis wiederum als Temperatur zurückgerechnet. Die Werte der einzelnen Simulationswerte sind in Tabelle \ref{tab:tempregler} gezeigt. 

    \begin{table}[hbt]
        \centering
        \caption{Parameter des Temperaturreglers}
        \label{tab:tempregler}
        \begin{tabular}{lr}\toprule
            Parameter   &   Wert\\\midrule
            \(K_1\)     & \SI{0.025}{\volt\per\celsius}\\\addlinespace
            \(K_2\)     & \SI{1,739}{\celsius\per\volt}\\\addlinespace
            \(u_{ein}\) & \SI{0.2}{\volt}\\
            \(u_{aus}\) & \SI{-0.2}{\volt}\\
            timeconst   & \SI{15}{\minute}\\\bottomrule
        \end{tabular}
    \end{table}
    
    Somit ist die Solltemperatur die Führungsgröße, die Eingangsspannung am Zweipunktegler entspricht der Stellgröße und die Ausgangstemperatur des Heizelements ist die Regelgröße. Der zeitliche Verlauf dieser Variablen über \SI{70}{\minute} ist in Abbildung \ref{fig:tempregler_plot} dargestellt. Durch Verändern der Schaltschwelle zu  \(\pm\SI{0.01}{\volt}\) erhält man Abbildung \ref{fig:tempregler_plot_new}.

    \begin{figure}
        \centering
        \includegraphics[width=\imagewidth]{../versuch1/tempregler_block.png}
        \caption{Blockschaltbild des Temperaturreglers}
        \label{fig:tempregler_block}
    \end{figure}    

    Aus den Abbildungen ist zu erkennen: Wird die Differenz zwischen Soll- und Ist-temperatur zu groß, wird die Schaltschwelle \(u_{ein}\) überschritten und der Zweipunktegler aktiviert die Heizung. Bei umgekehrten Vorzeichen wird die Heizung ausgeschalten. Durch Verkleinern der Schaltschwelle wird der Regler häufiger umschalten und die Ist-Temperatur weicht weniger von der Führungsgröße ab. Der Code zum Darstellen der Ergebnisse findet sich in \ref{lst:tempregler}.

    \begin{figure}
        \centering
        \begin{subfigure}{0.49\columnwidth}
            \includegraphics[width=1.0\columnwidth]{../versuch1/tempregler_plot.png}
            \caption{Schaltschwelle \(\pm\SI{0.2}{\volt}\)}
            \label{fig:tempregler_plot}   
        \end{subfigure}%
        \hfill%
        \begin{subfigure}{0.49\columnwidth}
            \includegraphics[width=1.0\columnwidth]{../versuch1/tempregler_plot_new.png}
            \caption{Schaltschwelle \(\pm\SI{0.01}{\volt}\)}
            \label{fig:tempregler_plot_new}
        \end{subfigure}
        \caption{Zeitliche Darstellung der Regelgrößen}
    \end{figure}

%-------------------------------------------------------------------------------------------------
%   Bibliography
%-------------------------------------------------------------------------------------------------

\printbibliography[heading=bibnumbered]

\section{Autorenbeiträge}
    Maileen Schwenk und Jan Hoegen erstellten die Vorbereitung und Messauswertung. Jan Hoegen schrieb den Bericht.

\section{Verfügbarkeit des Codes}
    Der Code zum Auswerten der Daten und Erstellen der Diagramme findet sich unter \url{https://github.com/JaxRaffnix/Regelungstechnik}. Ebenfalls ist hier der Code zum Erstellen dieser Ausarbeitung hinterlegt.

%-------------------------------------------------------------------------------------------------
%   Appendix
%-------------------------------------------------------------------------------------------------

\appendix

\section{MATLAB-Code der Sinussignale}
    \lstinputlisting[language=MATLAB, label={lst:sinus}]{../versuch1/sinus.m}

\section{MATLAB-Code zum Tiefpass}
    \lstinputlisting[language=MATLAB, label={lst:tiefpass}]{../versuch1/hochpass.m}

\section{MATLAB-Code zum Temperaturregler}
    \lstinputlisting[language=MATLAB, label={lst:tempregler}]{../versuch1/tempregler.m}

\end{document}