% !TeX TXS-program:bibliography = txs:///biber

\documentclass[
    paper=a4,
    fontsize=10pt,
    DIV=13,
    % twocolumn,
    oneside,
]{scrartcl}

%-------------------------------------------------------------------------------------------------
%   Packages, Style Guide
%-------------------------------------------------------------------------------------------------

\input{preamble}

%-------------------------------------------------------------------------------------------------
%   Title Page
%-------------------------------------------------------------------------------------------------

\labor{1}
\date{\today}

%-------------------------------------------------------------------------------------------------
%   Begin
%-------------------------------------------------------------------------------------------------

\begin{document}

\maketitle

%-------------------------------------------------------------------------------------------------
%   Abstract
%-------------------------------------------------------------------------------------------------

% \begin{abstract}
%     \noindent    
%     \subsubsection*{Abstract}
%     \blindtext
% \end{abstract}

%-------------------------------------------------------------------------------------------------
%   Text
%-------------------------------------------------------------------------------------------------

\section{Darstellung von Sinussignalen}
    Die Funktionen aus der Versuchsanleitung \cite{versuch1} werden mit MATLAB simuliert und in Abbildung \ref{fig:sinus} dargestellt.

    \begin{align}
        x_1(t) &= 2 \cdot sin(2\pi \cdot \SI{2}{\kilo\hertz} \cdot t)\\
        x_1(t) &= 2 \cdot sin(2\pi \cdot \SI{6}{\kilo\hertz} \cdot t - \frac{\pi}{4})
    \end{align}

    Darüber hinaus wird das zusammen gesetzte Signal \(x_3(t)=x_1(t) \cdot x_1(t)\) sowie eine Lissajous-Figur mit \(x_1(t)\) auf der x-Achse und \(x_2(t)\) auf der y-Achse abgebildet. Es ist zu erkennen, dass die Frequenz das doppelte von \(x_1(t)\) mit einem DC-Offset beträgt. Der Code zum Erstellen der Grafiken ist in Anhang \ref{lst:sinus} zu sehen.

    \begin{figure}[hbt]
        \centering
        \includegraphics[width=\imagewidth]{../versuch1/sinus}
        \caption{Darstellung der Sinussignale}
        \legend{Darstellung mit \(10^3\) Abtastpunkten}
        \label{fig:sinus}
    \end{figure}

    \subsection{Fehlerhafte Darstellungen der Lissajous-Figur}
        Wird der Zeitbereich auf \SIrange{0}{3}{\second} gelegt und somit die Größenordnung um \(10^3\) erhöht, ist die Figur zur Abbildung \ref{fig:sinus} gleich. Wird der Zeitbereich auf leicht verschoben, entsteht ein nicht interpretierbares Bild. Diese Effekte sind durch den Aliasing-Effekt zu begründen.
        Beide Änderungen sind in Abbildung \ref{fig:lissajous} gezeigt.

        \begin{figure}[hbt]
            \centering
            \includegraphics[width=\imagewidth]{../versuch1/lissjaou}
            \caption{Fehlerhafte Lissajous-Figuren}
            \label{fig:lissajous}
        \end{figure}

\section{Tiefpassanalyse}
    Für einen Tiefpass erster Ordnung gilt:
    \begin{align}
        \label{eq:tp}
        \frac{U_a}{U_e} = \frac{1}{1+jwRC}
    \end{align}

    Die Bauteilwerte mit einer Grenzfrequenz von \SI{100}{\kilo\hertz} und einem gewählten Kondensator \(C\) von \SI{1e-9}{\farad} berechnen sich zu:
    \begin{align}
        f_g &= \frac{1}{2 \pi RC} \overset{!}{=} \SI{1e5}{\hertz}\\
        \Rightarrow R&= \frac{1}{2 \pi \cdot f_g \cdot C }= \SI{1591.55}{\ohm}
    \end{align}

    Das Bodediagramm ist in Abbildung \ref{fig:bode} und die zugehörige Ortskurve in Abbildung \ref{fig:ortskurve} dargestellt. Da die Ortskurve achsensymmetrisch zur x-Achse ist, kann das Diagramm ohne den Verlust von Informationen um genau diese Spiegelung verkürzt werden. In MATLAB wird dies durch die Option \verb|ShowFullContour='off'| des \verb|nyquistplot|-Befehls erreicht. Der Code zum Erstellen der Diagramme findet sich in Anhang \ref{lst:tiefpass}.

    \begin{figure}
        \centering
        \includegraphics[width=\imagewidth]{../versuch1/bode}
        \caption{Bodediagramm des Tiefpasses}
        \label{fig:bode}
    \end{figure}

    \begin{figure}
        \centering
        \includegraphics[width=\imagewidth]{../versuch1/ortskurve}
        \caption{Ortskurven des Tiefpasses}
        \label{fig:ortskurve}
    \end{figure}

\section{Temperaturregler}
    Nun wird ein Temperaturregler in Abbildung \ref{temp_regler_schaltbild} simuliert. Der Zeitverlauf der eingestellten Solltemperatur, der Stellgröße und der Ausgangsgröße sind in Abbildung \ref{fig:temp_regler} dargestellt. Es ist zu erkennen, dass bei eingeschalteten Heizelement die Temperatur im Backofen schnell steigt bis zur Zieltemperatur von \SI{160}{\celsius}. Anschließend wird periodisch auf- und abgewärmt, bis die Führungsgröße auf \SI{0}{\celsius} verändert wird und die Temperatur absinkt. Die Parameter wurden im Anhang \ref{lst:temp_regler} definiert.

    \begin{figure}
        \centering
        % \includegraphics[width=1\imagewidth]{../versuch1/temp_regler_schaltbild.pdf}
        \caption{Blockschaltbild des Temperaturreglers}
        \label{fig:temp_regler_schaltbild}
    \end{figure}    

    \begin{figure*}
        \centering
        % \includegraphics[width=\imagewidth]{../versuch1/temp_regler.pdf}
        % \includesvg[width=\imagewidth, inkscapelatex=false]{../versuch1/temp_regler.svg}
        \caption{Zeitsignal der Regelgrößen}
        \label{fig:temp_regler}
    \end{figure*}

%-------------------------------------------------------------------------------------------------
%   Biblography
%-------------------------------------------------------------------------------------------------

\printbibliography[heading=bibnumbered]

\section{Autorenbeiträge}
    Maileen Schwenk und Jan Hoegen erstellten die Vorbereitung und Messauswertung. Jan Hoegen schrieb das Protokoll.

\section{Verfügbarkeit des Codes}
    Der Code zum Auswerten der Daten und Erstellen der Diagramme findet sich unter \url{https://github.com/JaxRaffnix/Regelungstechnik}. Ebenfalls ist hier der Code zum Erstellen dieser Ausarbeitung hinterlegt.

%-------------------------------------------------------------------------------------------------
%   Appendix
%-------------------------------------------------------------------------------------------------

\appendix

\section{MATLAB-Code der Sinussignale}
    \lstinputlisting[language=MATLAB, label={lst:sinus}]{../versuch1/sinus.m}

\section{MATLAB-Code zum Tiefpass}
    \lstinputlisting[language=MATLAB, label={lst:tiefpass}]{../versuch1/tiefpass.m}

\section{MATLAB-Code zum Temperaturregler}
    \lstinputlisting[language=MATLAB, label={lst:temp_regler}]{../versuch1/temp_regler.m}

\end{document}